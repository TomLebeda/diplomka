\subsection{Hodnotící algoritmus}
Za předpokladu, že je k dispozici referenční popis vytvořený expertem a testovaný popis získaný automaticky z přirozené řeči,
je potřeba tyto dva popisy nějakým způsobem porovnat a určit, do jaké míry odpovídá testovaný popis referenčnímu.

Právě to je předmětem této kapitoly: navrhnout algoritmus, který by zajišťoval porovnání testovaného popisu s referenčním a jehož výstupem by bylo nějaké hodnocení.

Základ hodnotícího algoritmu vychází ze struktury referenčního a testovaného popisu.
Oba typy se skládají z množiny objektů $\mathcal O$, množiny atributů $\mathcal A$ a množiny vazeb $\mathcal V$.
Dolním indexem je značeno, zda se jedná o množinu z referenčního popisu ($R$), nebo testovaného popisu ($T$).
Oba popisy obrázku lze tedy vyjádřit jako trojice:
\todo{definovat symbol pro popis obrázku? použít pro definice jiný symbol než =?}
\begin{align*}
	\text{\emph{referenční popis obrázku}} & = \bigl( \mathcal{O}_{R}, \mathcal{A}_{R}, \mathcal{V}_{R} \bigr) \\
	\text{\emph{testovaný popis obrázku}}  & = \bigl( \mathcal{O}_{T}, \mathcal{A}_{T}, \mathcal{V}_{T} \bigr)
\end{align*}
Porovnání a hodnocení podobnosti popisů by tak mohlo být převedeno na otázku porovnání podobnosti množin.
Otázky teorie množin jsou ale nad rámec zadání a pro tuto práci byl definován hodnotící algoritmus,
který je inspirován ztrátovými funkcemi.

Na obecné úrovni by se dalo říci, že hodnotící algoritmus počítá celkové ztráty pro chybějící objekty,
chybějící atributy, atributy s chybnou hodnotou, chybějící vazby mezi objekty.

Vstupem hodnotícího algoritmu jsou tedy oba popisy (referenční a testovaný) a k tomu ještě ztrátová tabulka, která určuje, jaký typ chyby způsobí jak velkou ztrátu.
Tuto ztrátovou tabulku také sestavuje expert, společně s referenčním popisem.

Výstupem hodnotícího algoritmu je pak množina označených číselných hodnot, které reprezentují celkové ztráty pro různé druhy chyb.

\newpage
\subsubsection{Chybějící objekty}
Pokud byl v referenčním popise expertem označený nějaký objekt,
který chybí v testovaném popise, předpokládá se, že expert považoval daný objekt za důležitý a uživatel jej nezmínil.

Uživatel si třeba nemusel objektu všimnout, nebo jej nepovažoval za dostatečně důležitý.
V obou případech se však jedná o nějaký rozpor se vzorovým popisem, který je třeba penalizovat.

Například pro situaci:
\begin{align*}
	\mathcal{O}_{R} & = \bigl\{\, \texttt{pes}, \texttt{veverka}, \texttt{houba}, \texttt{strom}\, \bigr\} \\
	\mathcal{O}_{T} & = \bigl\{\, \texttt{pes}, \texttt{veverka}\, \bigr\}
\end{align*}
je zřejmé, že v testovaném popisu chybí dva objekty: \texttt{houba} a \texttt{strom}.

Lze očekávat, že různé objekty budou ve scéně různě důležité a bylo by vhodné, aby byl tento fakt zohledněn při výpočtu ztráty.
Proto lze přiřadit objektům \emph{tagy}.
Jako \emph{tag} je v tomto kontextu chápána nějaká značka, která říká, že objekt patří do dané skupiny.
Například objekt \enquote{\texttt{tričko}} může mít přiřazen tag \enquote{\emph{oblečení}} a objekt \enquote{\texttt{pes}}
může mít přiřazen tag \enquote{\emph{zvíře}}.

Pro větší variabilitu systému bylo dále rozhodnuto, že každý objekt může mít přiřazený libovolný počet tagů.
Název tagů a jejich přiřazení objektům je volbou experta a je součástí tvorby referenčního popisu.
Ve ztrátové tabulce pak může expert definovat ztrátové hodnoty pro jednotlivé tagy a tím tak přepsat hodnotu ztráty na daném objektu.

Pokud má objekt přiřazeno více tagů, na jejich pořadí záleží.
Při určování ztrátové hodnoty pro daný objekt algoritmus postupně prochází jeden tag po druhém a kontroluje, zda je tento tag uvedený ve ztrátové tabulce.
Pokud ano, je ztráta na daném jednom chybějícím objektu rovna této hodnotě a další tagy na tomto objektu již nejsou kontrolované.
Pokud žádný z tagů není nalezen ve ztrátové tabulce, je použita výchozí hodnota ztráty pro jakýkoli chybějící objekt.

Pro názornost je schéma jednoho ukázkového případu na Obrázku~\ref{fig:example_missing_objects}.\todo{vyměnit obecné názvy za konkrétní?}
\begin{figure}[H]
	\centering
	\begin{tikzpicture}
		\node[draw, minimum width=4cm, minimum height=2cm] (comp) at (0, 0) {\parbox{3cm}{\small \centering výpočet ztráty pro chybějící objekty}};
		\node[anchor=west, inner sep=0mm] (out) at ($(comp.east) + (2.5, 0)$) {
			\small
			\centering
			\begin{tblr}{
					colspec={|c|c|},
					stretch={0.5},
					row{1}={abovesep={2mm},belowsep={2mm}},
					row{2}={abovesep={2mm}},
					row{9}={belowsep={2mm}},
				}
				\hline
				\textbf{Objekt} & \textbf{Ztráta} \\
				\hline
				\texttt{obj\_1} & 1.0             \\
				\texttt{obj\_2} & 0.5             \\
				\texttt{obj\_3} & 1.0             \\
				\texttt{obj\_4} & 0.0             \\
				\texttt{obj\_5} & 0.5             \\
				\texttt{obj\_6} & 0.5             \\
				\texttt{obj\_7} & 3.0             \\
				\texttt{obj\_8} & 0.0             \\
				\hline
			\end{tblr}
		};
		\node[anchor=south, inner sep=0mm] (ztr) at ($(comp.north) + (0, 1)$) {
			\small
			\centering
			\begin{tblr}{
					colspec={|c|c|},
					stretch={0.5},
					row{1}={abovesep={2mm},belowsep={2mm}},
					row{2}={abovesep={2mm}},
					row{4}={belowsep={2mm}},
				}
				\hline
				\textbf{Typ}    & \textbf{Ztráta} \\
				\hline
				\emph{výchozí}  & 1.0             \\
				\texttt{tag\_1} & 0.5             \\
				\texttt{tag\_3} & 3.0             \\
				\hline
			\end{tblr}
		};
		\node[anchor=south east, inner sep=0mm] (ref) at ($(comp.west) + (-1, 1)$) {
			\small
			\centering
			\begin{tblr}{
					colspec={|c|c|},
					stretch={0.5},
					row{1}={abovesep={2mm},belowsep={2mm}},
					row{2}={abovesep={2mm}},
					row{9}={belowsep={2mm}},
				}
				\hline
				\textbf{Objekt} & \textbf{Tagy}                    \\
				\hline
				\texttt{obj\_1} & ---                              \\
				\texttt{obj\_2} & \texttt{tag\_1}                  \\
				\texttt{obj\_3} & \texttt{tag\_2}                  \\
				\texttt{obj\_4} & \texttt{tag\_3}                  \\
				\texttt{obj\_5} & \texttt{tag\_1}, \texttt{tag\_2} \\
				\texttt{obj\_6} & \texttt{tag\_1}, \texttt{tag\_3} \\
				\texttt{obj\_7} & \texttt{tag\_3}, \texttt{tag\_1} \\
				\texttt{obj\_8} & \texttt{tag\_2}, \texttt{tag\_1} \\
				\hline
			\end{tblr}
		};
		\draw[-Stealth, rounded corners=2mm] (ref.south) |- node[near end, above] {\scriptsize referenční popis} ($(comp.west) + (0, 0.2)$);
		\draw[-Stealth] (ztr.south) -- node[midway, right] {\scriptsize ztrátová tabulka} (comp.north);
		\draw[-Stealth] (comp) -- node[midway, above] {\scriptsize vypočtené ztráty} (out);
		\node[draw] (in2) at ($(ref.south) + (0, -2.5)$) {$\mathcal{O}_{T} = \bigl\{ \, \texttt{obj\_{4}, \texttt{obj\_8}} \, \bigr\}$};
		\draw[-Stealth, rounded corners=2mm] (in2.north) |- node[near end, below] {\scriptsize testovaný popis} ($(comp.west) + (0, -0.2)$);
	\end{tikzpicture}
	\caption{Jednoduchá ukázka ztráty na chybějících objektech}\label{fig:example_missing_objects}
\end{figure}
Na příkladu zobrazeném na Obrázku~\ref{fig:example_missing_objects} je možné vidět, že objekt \texttt{obj\_4},
který byl nalezený v testovaném popisu, nezpůsobil žádnou ztrátu.
Ostatní objekty, které v testovaném popise chybí, způsobily ztrátu odpovídající hodnotě ze ztrátové tabulky.
Dále je možné si všimnout, že objekty \texttt{obj\_6} a \texttt{obj\_7} mají stejné tagy, ale v jiném pořadí, proto jsou jimi způsobené ztráty rozdílné.

Pro větší kompaktnost výstupu nejsou vypočtené ztráty prezentované pro každý objekt samostatně, ale jsou sečtené do finálního výsledku.
Tento proces sčítání ztrátových hodnot je přes všechny objekty, ale také jsou počítané hodnoty přes jednotlivé tagy.
Pokud bychom vzali tabulku vypočtených ztrát pro jednotlivé objekty z příkladu na Obrázku~\ref{fig:example_missing_objects},
šel by tento finální krok znázornit Obrázkem~\ref{fig:example_missing_objects_final_sum}.

\begin{figure}[ht!]
	\centering
	\begin{tikzpicture}
		\node[anchor=west, inner sep=0mm] (in) at (0, 0) {
			\small
			\centering
			\begin{tblr}{
					colspec={|c|c|c|},
					stretch={0.5},
					row{1}={abovesep={2mm},belowsep={2mm}},
					row{2}={abovesep={2mm}},
					row{9}={belowsep={2mm}},
				}
				\hline
				\textbf{Objekt} & \textbf{Tagy}                    & \textbf{Ztráta} \\
				\hline
				\texttt{obj\_1} & ---                              & 1.0             \\
				\texttt{obj\_2} & \texttt{tag\_1}                  & 0.5             \\
				\texttt{obj\_3} & \texttt{tag\_2}                  & 1.0             \\
				\texttt{obj\_4} & \texttt{tag\_3}                  & 0.0             \\
				\texttt{obj\_5} & \texttt{tag\_1}, \texttt{tag\_2} & 0.5             \\
				\texttt{obj\_6} & \texttt{tag\_1}, \texttt{tag\_3} & 0.5             \\
				\texttt{obj\_7} & \texttt{tag\_3}, \texttt{tag\_1} & 3.0             \\
				\texttt{obj\_8} & \texttt{tag\_2}, \texttt{tag\_1} & 0.0             \\
				\hline
			\end{tblr}
		};
		\node[anchor=west, inner sep=0mm] (out) at (7, 0) {
			\small
			\centering
			\begin{tblr}{
					colspec={|l|c|},
					stretch={0.8},
					row{1}={abovesep={2mm},belowsep={2mm}},
					row{2}={abovesep={2mm}},
					row{5}={belowsep={2mm}},
				}
				\hline
				\textbf{Chybějící objekty} & \textbf{Ztráta} \\
				\hline
				všechny                    & 6.5             \\[-1mm]
				objekty s \texttt{tag\_1}  & 4.5             \\[-1mm]
				objekty s \texttt{tag\_2}  & 1.5             \\[-1mm]
				objekty s \texttt{tag\_3}  & 3.5             \\
				\hline
			\end{tblr}
		};
		\draw[-Stealth] ($(in.east) + (0.3, 0)$) -- ($(out.west) + (-0.3, 0)$);
	\end{tikzpicture}
	\caption{Finálního krok při počítání ztrát na chybějících objektech}\label{fig:example_missing_objects_final_sum}
\end{figure}

Zde je vhodné zmínit, že popsaný algoritmus výpočtu ztráty je pouze jednou z možností.
Pro konkrétní nasazení by mohla být vhodnější jiná forma počítání ztrátových hodnot,
například vypočítat průměr, sumu nebo maximum či minimum.
Tento způsob počítání ztráty byl zvolen proto, že poskytuje expertovi další stupeň volnosti,
jak vyjádřit důležitost objektů.

\subsubsection{Chybějící atributy}
Pokud expert popsal na daném objektu nějakou vlastnost a odpovídající atribut chybí v testovaném popise,
je zde stejný předpoklad jako u chybějících objektů a také se jedná o chybu, která způsobí navýšení ztrátové hodnoty.

Způsob hodnocení je obdobný jako pro chybějící objekty, pouze s tím rozdílem, že atributům nejsou přiřazovány žádné tagy.
Tagy u objektů sloužily k detailnějšímu určení důležitosti objektu a později ke sčítání ztrátových hodnot,
v případě atributů tuto funkci zastává samotný název atributu.

Například pokud by byl libovolný chybějící atributu penalizován ztrátou $1.0$, pak by bylo možné
specifikovat, že vynechání atributu \emph{\enquote{barva}} je méně závažné než všechny ostatní a penalizovat pouze ztrátou $0.5$.
Naopak vynechání atributu \emph{\enquote{výraz v obličeji}} (např.~úsměv nebo zamračení) by mohl být považován za důležitý a expert by
mohl ve ztrátové tabulce definovat, že jeho vynechání bude penalizováno ztrátou $2.0$.
\todo{dát konkrétní příklad včetně tabulek, výpočtů \ldots ?}

\subsubsection{Chybějící vazby mezi objekty}
Posledním typem entity, která může chybět při porovnávání referenčního a testovaného popisu, jsou vazby mezi objekty popsané \emph{triplety}.

Zde je hodnotící algoritmus velmi podobný jako při počítání ztráty pro chybějící objekty.
Rozdíl zde spočívá v tom, že místo tagů jsou zde použité samotné názvy vazeb, podobně jako u chybějících atributů jsou použité názvy atributů.

Expert ve ztrátové tabulce opět specifikuje, jaká je výchozí hodnota ztráty, pokud v testovaném popise bude chybět libovolná vazba mezi objekty.
Následně může také specifikovat, že některé konkrétní typy vazeb (podle jejich názvu) mají větší či menší důležitost a tak jimi způsobená ztráta může být větší nebo menší.
Algoritmus pak při výpočtu použije přednostně takovou hodnotu, která je více specifická.
\todo{dát konkrétní příklad včetně tabulek, výpočtů \ldots ?}

\subsubsection{Atributy s chybnou hodnotou}
Pokud v byl v testovém popise nalezen atribut, který odpovídá objektem a názvem nějakému atributu z referenčního popisu,
ale neodpovídá svou hodnotou žádnému referenčnímu atributu, pak se předpokládá, že uživatel udělal chybu při popisu a řekl špatnou hodnotu.

Příkladem takové neshody by mohlo být:
\begin{align*}
	\mathcal{A}_{R} & = \bigl\{ \, \texttt{tričko: barva = modrá } \, \bigr\} \\
	\mathcal{A}_{T} & = \bigl\{ \, \texttt{tričko: barva = zelená} \, \bigr\}
\end{align*}
pak je možné předpokládat, že uživatel udělal při popisu chybu a řekl špatnou barvu.

Způsob počítání ztrátové hodnoty z těchto chybných hodnot je v principu stejný jako u chybějících objektů a atributů.
Expert ve ztrátové tabulce definuje výchozí hodnotu a k tomu případně specifické případy, kdy je tato ztrátová hodnota jiná.

% Expert ve ztrátové tabulce definuje, jaká je výchozí ztráta pro chybně určenou hodnotu atributu a případně jaké konkrétní situace tuto hodnotu přepisují na jinou.
Rozdíl oproti předchozím výpočtům ovšem spočívá v tom, že různé hodnoty si mohou být různě blízko.
Například pokud uživatel zamění hnědou a oranžovou barvu veverky, tak se pravděpodobně ve většině případů bude jednat o menší chybu,
než kdyby oranžovou zaměnil třeba za zelenou.

Z toho důvodu je ztrátová tabulka navržena tak, aby expert mohl definovat výchozí ztrátu pro všechny chybné atributy,
výchozí ztrátu pro konkrétní atribut bez ohledu na hodnoty, ztrátu pro konkrétní atribut a zároveň množinu konkrétních hodnot, jejichž záměna tuto ztrátu způsobí

Pro lepší názornost následuje příklad, který předpokládá, že expert definoval ztrátovou tabulku shodnou s Tabulkou~\ref{tab:example_wrong_attrs_values}.
Tu je možné číst následujícím způsobem:
\begin{enumerate}
	\item Libovolný chybný atribut způsobí ztrátu 1.0, pokud není ve zbytku tabulky definována více specifická alternativa.
	\item Pokud je chybná hodnota v atributu \enquote{\texttt{barva}}, pak je způsobená ztráta rovna 0.5, pokud se nejedná o záměnu některých konkrétních hodnot:
	      \begin{itemize}
		      \item Pokud je zaměněna barva \enquote{\texttt{žlutá}} a \enquote{\texttt{oranžová}}, pak je ztráta 0.3.
		      \item Pokud je zaměněna barva \enquote{\texttt{červená}}, \enquote{\texttt{modrá}} nebo \enquote{\texttt{zelená}} (libovolná dvojice), pak je ztráta rovna 0.8.
	      \end{itemize}
	\item Pokud je chybná hodnota v atributu \enquote{\texttt{tvar}}, pak je ztráta 1.5.
\end{enumerate}
Ukázka některých konkrétních atributů s chybnými hodnotami a k nim vypočtených ztrát je zobrazena v Tabulce~\ref{tab:example_wrong_attrs_table}.

\begin{table}[H]
	\centering
	\begin{tblr}{
			colspec={|c|l|},
			row{3}={abovesep={2.5mm},belowsep={2.5mm}},
			row{4}={abovesep={2.5mm},belowsep={2.5mm}}
		}
		\hline
		\textbf{Atribut} & \textbf{Ztráta}                                                         \\
		\hline
		\emph{výchozí}   & 1.0                                                                     \\
		\hline
		\texttt{barva}   &
		\begin{tblr}{colspec={|l|l|}, stretch=0.7, row{2}={abovesep={1mm}}}
			\hline
			\textbf{Hodnoty}                                  & \textbf{Ztráta} \\
			\hline
			\emph{výchozí}                                    & 0.5             \\
			\texttt{žlutá}, \texttt{oranžová}                 & 0.3             \\
			\texttt{červená}, \texttt{modrá}, \texttt{zelená} & 0.8             \\
			\hline
		\end{tblr} \\
		\hline
		\texttt{tvar}    &
		{
				\centering
				\begin{tblr}{colspec={|l|l|}, stretch=0.7, row{2}={abovesep={1mm}}}
					\hline
					\textbf{Hodnoty} & \textbf{Ztráta} \\
					\hline
					\emph{výchozí}   & 1.5             \\
					% \texttt{žlutá}, \texttt{oranžová}                 & 0.3             \\
					% \texttt{červená}, \texttt{modrá}, \texttt{zelená} & 0.8             \\
					\hline
				\end{tblr}
		}                                                                                          \\
		\hline
	\end{tblr}
	\caption{Část ukázkové ztrátové tabulky}\label{tab:example_wrong_attrs_table}
\end{table}

\begin{table}[H]
	\centering
	\begin{tblr}{colspec={|l|l|l|}, stretch={0.7}, row{1}={abovesep={2mm},belowsep={1mm}}, row{2}={abovesep={1mm}}}
		\hline
		\textbf{Referenční atribut}      & \textbf{Testovaný atribut}      & \textbf{Ztráta} \\
		\hline
		\texttt{pes: barva = hnědá}      & \texttt{pes: barva = hnědá}     & 0.0             \\
		\texttt{oheň: barva = žlutá}     & \texttt{oheň: barva = oranžová} & 0.3             \\
		\texttt{jablko: barva = červená} & \texttt{jablko: barva = modrá}  & 0.8             \\
		\texttt{tričko: barva = modrá}   & \texttt{tričko: barva = zelená} & 0.8             \\
		\texttt{čepice: barva = modrá}   & \texttt{čepice: barva = žlutá}  & 0.5             \\
		\texttt{penál: tvar = válec}     & \texttt{penál: tvar = kvádr }   & 1.5             \\
		\texttt{pes: činnost = leží}     & \texttt{pes: činnost = leží }   & 1.0             \\
		\hline
	\end{tblr}
	\caption{Výpočtu ukázkových ztrát chybných hodnot atributů podle Tabulky~\ref{tab:example_wrong_attrs_table}}\label{tab:example_wrong_attrs_values}
\end{table}

Přirozeně se naskýtá otázka toho, jak počítat ztrátu při konfliktu objektů se stejnými atributy a nebo objekty s více atributy.
Pro řešení těchto konfliktů bylo rozhodnuto, že hodnotící algoritmus bude optimistický a bude uvažovat nejmenší možnou chybu,
pokud je podle ztrátové tabulky k dispozici více možných interpretací.

\subsubsection{Výstup hodnotícího algoritmu}
Finálním výstupem hodnotícího algoritmu je množina označených číselných hodnot, které jsou akumulované ztráty pro různé druhy chyb.
Přirozenou datovou strukturou, která by odpovídala formátu, by bylo asociativní pole (někdy také označované jako hash-tabulka).

Výstup hodnotícího algoritmu, jak byl popsán v předchozích částech, obsahuje vždy celkovou ztrátu pro daný typ chyby
a poté akumulované ztrátové hodnoty stejného typu, ale rozdělené ještě podle jednotlivých kategorií.
\todo{příklad?}
Konkrétně bude výstupní struktura obsahovat:
\begin{itemize}
	\item celkovou ztrátu způsobenou všemi chybějícími objekty
	\item celkovou ztrátu způsobenou všemi chybějícími atributy
	\item celkovou ztrátu způsobenou všemi chybějícími vazbami mezi objekty
	\item celkovou ztrátu způsobenou všemi atributy s chybnými hodnotami
	\item ztráty způsobené chybějícími objekty s konkrétním tagem
	\item ztráty způsobené chybějícími atributy s konkrétním názvem/typem atributu
	\item ztráty způsobené chybějícími vazbami s konkrétním názvem/typem vazby
	\item ztráty způsobené atributy s chybnými hodnotami, rozdělené podle názvu/typu atributu
\end{itemize}

Je vhodné zmínit, že při návrhu hodnotícího algoritmu bylo předpokládáno, že pro různé konkrétní nasazení mohou být různé požadavky na formu výstupu.
V rámci implementace popsané později existuje proto \enquote{skrytý} mezi-krok, ve kterém jsou k dispozici všechny jednotlivé ztráty pro konkrétní objekty, atributy a vazby,
ze kterých je poté počítána právě popsaná výsledná struktura.
Tento mezi-výsledek by bylo možné výhodně použít pro implementaci alternativních způsobů hodnocení, nebo případně rovnou použít jako výstup,
pokud by daná aplikace benefitovala z podrobnější analýzy výsledků.
