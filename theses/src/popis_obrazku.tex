\clearpage
\section{Korespondence obrazu a jeho slovního popisu}
Korespondence mezi obrazem a jeho popisem je problematika řešená v řadě reálných praktických aplikací.
Tato kapitola popisuje základní principy práce s obrazem a spojitost s jeho popisem, a některé vybrané příbuzné problémy.
Na závěr jsou v této kapitole zmíněné rozdíly v přístupu, který byl zvolen pro tuto práci.

\subsection{Práce s obrazovými daty}
Popis obrázku je v kontextu výpočetní technologie a umělé inteligence rozšířené téma, které nachází uplatnění v mnoha praktických aplikacích.
Jedná se o jednu z mnoha problematik, které se obecně v úlohách počítačového zpracování obrazových dat řeší.

Úlohy, které se týkající zpracování obrazu, lze podle učebního textu~\cite{conv-gruber} rozdělit do čtyř základních typů:
\begin{enumerate}
	\item \textbf{Klasifikace}\\
	      Jedná se o úlohu přiřazování obrázku k nějaké třídě.
	      Množina tříd, do kterých jsou obrázky přiřazované, jsou často známé předem a může jich být různá množství,
	      od několika jednotek pro úzce specializované úlohy, například v medicíně, nebo pro obecnější úlohy u tisíce.

	      V úlohách klasifikace není brán ohled na prostorové informace, celý obrázek je chápán jako jeden celek,
	      jemuž je přiřazena příslušnost do vybrané třídy.
	      Typickým příkladem může být rozdělení obrázku do kategorií \emph{kočka} nebo \emph{pes} podle toho,
	      jaké zvíře je na obrázku znázorněno.
	\item \textbf{Sémantická segmentace}\\
	      Sémantická segmentace obrázků spočívá v přiřazení nějaké ze tříd každému pixelu.
	      Tím dojde k rozdělení obrázku do disjunktních oblastí, které jsou (sémanticky) označené danou třídou.

	      V úlohách tohoto typu se již patrně zohledňuje prostorová informace jednotlivých částí,
	      avšak jednotlivé instance různých objektů pod společnou třídou rozlišené nejsou.
	      Příkladem typické segmentace může být určení toho, které pixely odpovídají obloze, které zemi, a které třeba člověku.

	      \newpage
	\item \textbf{Detekce objektů}\\
	      Principem detekce objektů je v obraze určit, zda se v něm nachází jeden nebo více hledaných objektů daného typu.
	      Vstupem bývá obrázek a požadavek na detekci nějakého typu objektu a výstupem může být obrázek s označením
	      objektů pomocí ohraničujících rámců (angl.~bounding-box) s přiřazeným označením třídy.

	      Tento typ úlohy může být vhodný například pro sledování (tracking) lidí či zvířat, pro následnou klasifikaci nebo jinou analýzu.

	\item \textbf{Segmentace instancí}\\
	      Tento typ úlohy představuje kombinaci předchozích typů úloh.
	      Princip spočívá v tom, že v obrázku jsou označené pixely odpovídající nějakému hledanému objektu stejně jako při
	      sémantické segmentaci, ovšem s tím rozdílem, že není nutné takto označit každý pixel v obrázku a naopak je
	      dbáno na rozlišení jednotlivých instancí daných objektů.

	      Výsledkem je tedy informace o tom, které pixely odpovídají jakému konkrétnímu objektu.
	      Například na obrázku, kde by bylo zobrazeno více psů a koček, by bylo cílem označit kde přesně se nachází
	      který pes a která kočka.
\end{enumerate}

Všechny výše zmíněné typy úloh, respektive jejich výstupy, lze považovat za popis obrázku v tom smyslu,
že výstupem je jistá forma informace o tom, co daný obraz zachycuje.
Nejedná se však přímo o hluboké či komplexní sémantické znalosti, spíše o \enquote{hrubé} zařazení do nějaké třídy nebo forma výpisu zobrazených objektů.

V současnosti jsou obrazová data zpracovávána převážně pomocí velkých neuronových sítí.
Díky pokrokům ve výpočetní síle a dostupnosti datových sad jsou tyto modely stále sofistikovanější a schopné řešit složitější úkoly.
Součástí tohoto pokroku je také schopnost modelů analyzovat obrazová data a v kombinaci se zpracováním přirozené řeči i
odpovídat na otázky týkající se komplexnější sémantiky.
{\footnotesize\color{red} [přidat zdroj, třeba na ten nový GPT-4o, kde je ukázka? Tady nemám přímo nic faktického, tak by to mohlo stačit? Tenhle odstavec je hlavně pro nějaké \enquote{napojení} popisu obrázků na sémantiku]}

\newpage
\subsection{Sémantický popis obrazu}
Jednou z oblastí, kde je využívána ontologie a sémantický popis obrázku, jsou systémy pro vyhledávání obrázků (angl.~image retrieval).

Například Sarwar et al.~\cite{SARWAR2013285} prezentuje metodu zabývající se zdokonalením přesnosti vyhledávání obrázků,
na základě vstupního dotazu v přirozeném jazyce.
Navrhuje metodu, která má tohoto cíle dosáhnout za využití doménové ontologie, lokální sémantické informace v podobě
obrazových deskriptorů a jejich kvalitativní prostorové relace.
% V rámci vyhledávání využívá pro práci s dotazy také dříve zmíněný RDF standard jako náhradu za běžně užívané postupy založené na klíčových slovech.
Navržená architektura v první fázi předzpracuje poskytnuté obrázky tak, že je rozdělí na jednotlivé sektory, kterým jsou
přiřazené značky reprezentující obecné, předem určené, koncepty, jako jsou například \enquote{tráva}, \enquote{nebe} nebo \enquote{písek}.
% Takto anotované obrázky je pak možné rozdělit do předem definovaných kategorií.
Další část architektury, kterou Sarwar et al.~\cite{SARWAR2013285} prezentuje, slouží k transformování uživatelského vstupu (dotazu)
do několika předem definovaných forem využívajících RDF triplety.
Poslední fází zpracování je pak využití doménové ontologie pro určení sémantické podobnosti dotazu a dostupných anotovaných obrázků,
ze kterého plyne výběr výstupního obrázku odpovídající vstupnímu dotazu.

Další příbuznou problematikou je automatická tvorba sémantických popisků k obrázkům, kterou
ve své práci prezentuje Meiyu et al.~\cite{img_semantic_descr_annotation}.
Využívá k tomu SIFT~\cite{sift} deskriptorů, kvůli jejich dobrému poměru mezi výpočetní náročností, informačním objemem a spolehlivostí.
Následně je v práci~\cite{img_semantic_descr_annotation} prezentováno, že díky internetu a z něj plynoucímu množství koexistujících
obrázků a jejich textových popisů, byl sestaven model, který je schopen mapovat extrahované příznakové vektory (angl.~feature vector)
ze SIFT deskriptorů na textové anotace, ze kterých jsou získané sémantické popisy.
Výsledky, které Meiyu et al.~\cite{img_semantic_descr_annotation} prezentuje, pak zobrazují obrázek,
k němuž jsou automaticky doplněné anotace objektů a konceptů, které se na něm vyskytují.

Doposud popisované problémy a jejich řešení využívají převážně neuronových sítí a statistických metod,
jejichž základním předpokladem je dostatečný počet a kvalita trénovacích dat.

Vzhledem k motivaci této práce, kterou je hodnocení kvality popisu daného obrázku na sémantické úrovni pro
usnadnění detekce kognitivních poruch, byl zvolen expertní přístup se zaměřením na
kvalitativní analýzu výsledků, průhlednost celého procesu s možností detailních úprav a reprodukovatelnost.


% Zde navržený a implementovaný systém se zaměřuje na hodnocení kvality popisu obrázku v přirozené řeči, 
% kde vstupní 
% předpokládá pečlivě vytvořená a anotovaná data, která lze využít opakovaně 

% 1. Popisy obrázků jsou nejčastější formou nějaký klasifikace nebo detekce, založený na deskriptorech (sifty, surfy, ...) nebo přes velký neuronovky

% 2. Sémantika se u obrázků řeší spíš formou vytváření popisků (generování textu) nebo opačně - generovat obrázek z textu

% 3. Přímo sémantika obrázku a srovnat se sémantikou textu jsou maximálně nějaký přiřazovačky (vyhledávače obrázků)
