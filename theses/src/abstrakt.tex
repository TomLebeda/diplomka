\section*{Abstrakt}
Zpracování přirozeného jazyka spolu se sémantickou analýzou a reprezentací znalostí jsou v dnešní době nedílnou součástí mnoha aplikací.
Tato práce se zabývá návrhem a~realizací systému pro určení sémantické korespondence obrázku a jeho popisu v přirozené řeči
za účelem usnadnění diagnostických testů pro detekci kognitivních poruch, s~důrazem na expertní přístup k problematice a z něj
plynoucí detailní kontrolu nad jednotlivými fázemi procesu.
V práci je popsán návrh systému a jeho částí, včetně zdůvodnění učiněných rozhodnutí.
Poté je popsána implementace jednotlivých částí a řešení problémů, které se během práce vyskytly.

\section*{Klíčová slova}
zpracování přirozeného jazyka, sémantická analýza, reprezentace znalostí, sémantická korespondence obrazu a řeči, sémantický popis obrazu,
bezkontextové gramatiky, porozumění řeči

\thispagestyle{empty}
\newpage

\section*{Abstract}
Natural language processing, along with semantic analysis and knowledge representation, has become an integral part of many contemporary applications.
This thesis focuses on the design and implementation of a system to determine the semantic correspondence between an image and its natural language description,
aimed at facilitating diagnostic tests for the detection of cognitive disorders.
Emphasis is placed on an expert approach to the subject, allowing for detailed control over the various stages of the process.
The thesis outlines the design of the system and its components,
including the rationale behind the decisions made.
Subsequently, the implementation of the individual components and the solutions to the problems encountered during the work are described.


\section*{Keywords}
natural language processing, semantic analysis, knowledge representation, semantic correspondence of image and speech, semantic image description,
context-free grammars, spoken language understanding
\thispagestyle{empty}
