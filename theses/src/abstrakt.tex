\section*{Abstrakt}
% Hlasoví asistenti jsou systémy, jejichž hlavním účelem je zjednodušit nebo automatizovat běžné každodenní činnosti.
% Uživatelé s~nimi interagují pomocí hlasu, což umožňuje jejich aplikace v~situacích, kdy má člověk obě ruce a/nebo oči plně zaměstnané, nebo má omezené zrakové či pohybové schopnosti.
% Tato práce se zabývá návrhem a~implementací systému osobní hlasové asistentky s~důrazem na snadnou modifikovatelnost a~rozšiřitelnost,
% schopnou základního ovládání vybrané komunikační platformy a~reakcí na vnější podněty.
% V~práci je popsán návrh architektury celého systému i~jednotlivých částí a~zdůvodněné některé designové volby.
% Poté je popsána implementace a~řešení problémů, které se během práce vyskytly.
Zpracování přirozeného jazyka spolu se sémantickou analýzou a reprezentací znalostí jsou v dnešní době nedílnou součástí mnoha aplikací.
Tato práce se zabývá návrhem a realizací systému pro určení sémantické korespondence obrázku a jeho popisu v přirozené řeči
za účelem usnadnění diagnostických testů pro detekci kognitivních poruch, s důrazem na expertní přístup k problematice a z něj
plynoucí detailní kontrolu nad jednotlivými fázemi procesu.
V práci je popsán návrh systému a jeho částí, včetně zdůvodnění učiněných rozhodnutí.
Poté je popsána implementace jednotlivých částí a řešení problémů, které se během práce vyskytly.

\section*{Klíčová slova}
Zpracování přirozeného jazyka, sémantická analýza, reprezentace znalostí, sémantická korespondence obrazu a řeči, sémantický popis obrazu,
bezkontextové gramatiky, porozumění řeči.

% hlasové dialogové systémy, osobní asistentka, zpracování přirozeného jazyka, porozumění řeči, řídicí pravidla, dialogový manažer, reprezentace znalostí, bezkontextové gramatiky
\thispagestyle{empty}
\newpage

\section*{Abstract}
Natural language processing, along with semantic analysis and knowledge representation, has become an integral part of many contemporary applications.
This thesis focuses on the design and implementation of a system to determine the semantic correspondence between an image and its natural language description,
aimed at facilitating diagnostic tests for the detection of cognitive disorders.
Emphasis is placed on an expert approach to the subject, allowing for detailed control over the various stages of the process.
The thesis outlines the design of the system and its components,
including the rationale behind the decisions made.
Subsequently, the implementation of the individual components and the solutions to the problems encountered during the work are described.

% Voice assistants are systems with main purpose to simplify or automatize common everyday tasks.
% Users interact with them through voice, allowing them to be used in situations where user has both hands and/or eyes fully occupied, or has limited visual or physical abilities.
% The topic of this thesis is design and implementation of personal voice assistant with emphasis on easy modifiability and extensibility,
% capable of basic management of the selected communication platform and reacting to external events.
% The thesis describes the design of the whole system as well as individual parts and reasons some design choices.
% The implementation is described along with problems that occurred during the work and their solutions.

\section*{Keywords}
Natural language processing, semantic analysis, knowledge representation, semantic correspondence of image and speech, semantic image description,
context-free grammars, spoken language understanding.
% Spoken dialog systems, personal assistant, natural language processing, spoken language understanding, dialog manager, state representation, context-free grammars
\thispagestyle{empty}
