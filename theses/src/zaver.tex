\section{Závěr}
% NOTE: co bylo cílem
Cílem této práce bylo navrhnout a implementovat systém pro hodnocení míry korespondence mezi
obrázkem a jeho popisem v přirozené řeči.
Motivací pro takovýto systém bylo jeho využití pro usnadnění diagnostických testů pro detekci kognitivních poruch.
Zvolené řešení bylo založeno na expertním přístupu, který byl pro toto použití shledán vhodným.

% NOTE: na co byl kladen důraz
Během návrhu a realizace byl kladen důraz na to, aby všechny důležité procesy a subsystémy umožňovaly
precizní kontrolu a možnost případných detailních úprav pro dosažení co nejlepších výsledků v reálném nasazení.
Dále bylo během návrhu dbáno na modularitu a obecnost, aby bylo možné jednotlivé procesy nahradit
za případné vhodnější alternativy pro konkrétní aplikace v různých oblastech.

% NOTE: jak dopadla architektura
Výsledná architektura celého systému se skládá z referenčního popisu obrázku, subsystému pro zpracování přirozené
řeči a extrakce sémantické informace, která je v ní obsažena, a hodnotícího algoritmu,
který porovná referenční popis se získanou sémantikou a na základě ztrátové tabulky vygeneruje
vektor hodnot, které reprezentují míru sémantické shody mezi vzorovým a testovaným popisem.
Tento výstupní vektor je pak možné použít jako výchozí bod pro další zpracování, například jako vektor příznaků pro klasifikační metody.
Je vhodné zdůraznit, že realizovaný způsob hodnocení umožňuje pracovat s různými úrovněmi detailu, od více obecných, až po velmi detailní.
Tato přizpůsobitelná granularita výstupu umožňuje precizní nastavení pro konkrétní potřeby využití v reálném nasazení.

% NOTE: popis procesu, výhody a nevýhody jednotlivých částí
% Celkový návrh systému je založený na expertním přístupu k řešení dané problematiky.
Od lidského experta v daném oboru je vyžadováno, aby vybral obrázek a sestavil jeho referenční popis, který udává,
jak by měl vypadat ideální popis vybraného obrázku.
Dále je nutné vytvořit gramatiku pro sémantické zpracování řeči a nakonec je na expertovi, aby vytvořil podle referenčního popisu
ztrátovou tabulku, pomocí které specifikuje závažnost různých druhů chyb v popisu.
Referenční popis i ztrátová tabulka mají podobu JSON souborů s určitým formátem, což představuje dobrý kompromis mezi čitelností člověkem a strojovou zpracovatelností.

% NOTE: jaké jsou omezení a nevýhody
Tento expertní přístup s sebou přináší omezení ve smyslu vysokých počátečních nákladů a náročné přípravy,
jelikož tvorba referenčního popisu, gramatik a ztrátové tabulky může představovat časově náročný úkol.
Na druhou stranu ale zvolený přístup umožňuje velmi detailní a precizní kontrolu nad jednotlivými procesy a fázemi, právě
díky tomu, že vstupní vzorová data mohou být pečlivě sestavena a nejsou generována automaticky pomocí statistických metod či neuronových sítí,
jejichž výsledky nemusí být dostačující pro takto specializovanou problematiku.

% NOTE: spgf - vedlejší produkt
V rámci implementace navržených systémů byl také navržen nový formát pro sémantické gramatiky, včetně implementace jeho parseru.
Jedná se o jednou z klíčových částí zpracování přirozené řeči a extrakci sémantické informace ve zde prezentovaném systému.
Tento nový formát, pojmenovaný \enquote{Semantic Parsing Grammar Format}, byl založen na existujícím standardu Speech Recognition Grammar Specification,
oproti kterému nabízí rozšíření funkčnosti, především ve smyslu ergonomie použití a možnostech volby z vícero parsovacích strategií.
Tento formát a k němu přidružený software je možné použít nezávisle na zbytku této práce i pro zcela jiné účely.

% NOTE: jaké byly komplikace
Během realizace se nevyskytly žádné významné technické komplikace.
Jediným výraznějším problémem byla absence expertních znalostí z oblasti medicíny zaměřující se na kognitivní poruchy,
která vedla k tomu, že vytvořený referenční popis obrázku a ztrátovou tabulku bylo možné brát pouze jako ilustrační a pro testovací účely.
To však ve výsledku nepředstavovalo výrazný problém, jelikož účelem práce bylo navrhnout systém pro výše popsaný účel,
na jehož otestování ilustrační vstupní data postačovala.

% NOTE: jak dopadly výsledky
Výsledky prezentované v kapitole~\ref{sec:vyhodnoceni} ukazují, že navržený systém je schopen extrahovat z~přirozeného
textu požadované sémantické informace, za předpokladu dobře navrženého referenčního popisu a především kvalitně sestavených gramatik.
Vzhledem k již popsaným problémům s absencí expertem poskytnutých dat bylo vyhodnocení porovnávacího a hodnotícího algoritmu
obtížnější, avšak dosažené výsledky nasvědčují tomu, že skutečně méně kvalitním popisům obrázku odpovídají vyšší hodnoty
ztrát, což je v~souladu s původním předpokladem.

% NOTE: jaká jsou potenciální budoucí rozšíření
Důkladnější testování hodnotícího algoritmu je plánováno v rámci již zmíněného projektu \projekt{},
který umožní detailnější analýzu nad větším množstvím dat a spolupráci s experty z daného oboru.
Dalším potenciálním rozšířením je otestování alternativních přístupů k extrakci sémantické informace,
například s využitím dnes populárních velkých jazykových modelů (angl.~large language models, LLMs).


