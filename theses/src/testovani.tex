\subsection{Testování na datech}
Posledním krokem bylo ověření funkčnosti navrženého a implementovaného systému.
K tomu bylo potřeba získat vhodná data, konkrétně obrázek a k němu přirozené popisy, pokud možnost v textové podobě.
Tato data byla získána z projektu
\todo{referovat DigiDiaDem data/projekt?}
\todo{jak moc rozebírat účel toho projektu a dat?}

Popisovanou scénou byla kresba zobrazená na Obrázku~\ref{fig:summer}.
K tomuto obrázku bylo získáno několik desítek přepisů, z nichž bylo několik náhodně vybráno pro účely testování.

Referenční popis daného obrázku, stejně jako SPGF gramatika a ztrátová tabulka,
byly vytvořené autorem této práce a nikoli expertem z oboru, odkud pocházejí nasbíraná data.
Hodnoty ve ztrátové tabulce je tedy třeba chápat jako ilustrační a ve výsledcích je možné srovnávat pouze relativní ztráty,
nelze dělat smysluplné závěry z konkrétních hodnot.
To však nepředstavuje problém, protože cílem zde bylo ověřit funkčnost navrženého systému, na což ilustrační hodnoty postačují.

Vytvořený referenční popis Obrázku~\ref{fig:summer} je možné prozkoumat spolu se sestavenou SPGF gramatikou u zdrojových kódů.
Ztrátová tabulka použitá při testování je na Výpisu~\ref{lst:loss_table_example}.

\begin{lstlisting}[
	% there are many more options of styling, see the official documentation, these are just the defaults I like
	frame=single, % make single-line frame around the verbatim
	framesep=2mm, % put some more spacing between the frame and text
	aboveskip=5mm, % put some more space above the box
	basicstyle={\linespread{0.9}\small\ttfamily}, % use typewriter (monospace) font
	caption={Ztrátová tabulka použitá při testování}, % set the caption text
	captionpos=b, % put the caption at the bottom (b) or top (t) or both (bt)
	label={lst:loss_table_example}, % label to be referenced via \ref{}
	numbers=left, % line numbers on the left
	numberstyle={\scriptsize\ttfamily\color{black!60}}, % the style for line numbers
	escapeinside={<@}{@>} % between those sequences are command evaluated
]
<@\textcolor[HTML]{FF1010}{\texttt{\{}}@>
<@\textcolor[HTML]{000000}{\texttt{\ \ }}@><@\textcolor[HTML]{255CFF}{\texttt{"missing\_objects"}}@><@\textcolor[HTML]{1041FF}{\texttt{:}}@><@\textcolor[HTML]{000000}{\texttt{\ }}@><@\textcolor[HTML]{DE6F10}{\texttt{3}}@><@\textcolor[HTML]{1041FF}{\texttt{,}}@>
<@\textcolor[HTML]{000000}{\texttt{\ \ }}@><@\textcolor[HTML]{255CFF}{\texttt{"missing\_attributes"}}@><@\textcolor[HTML]{1041FF}{\texttt{:}}@><@\textcolor[HTML]{000000}{\texttt{\ }}@><@\textcolor[HTML]{DE6F10}{\texttt{1}}@><@\textcolor[HTML]{1041FF}{\texttt{,}}@>
<@\textcolor[HTML]{000000}{\texttt{\ \ }}@><@\textcolor[HTML]{255CFF}{\texttt{"missing\_triplets"}}@><@\textcolor[HTML]{1041FF}{\texttt{:}}@><@\textcolor[HTML]{000000}{\texttt{\ }}@><@\textcolor[HTML]{DE6F10}{\texttt{2}}@><@\textcolor[HTML]{1041FF}{\texttt{,}}@>
<@\textcolor[HTML]{000000}{\texttt{\ \ }}@><@\textcolor[HTML]{255CFF}{\texttt{"numberless\_penalty"}}@><@\textcolor[HTML]{1041FF}{\texttt{:}}@><@\textcolor[HTML]{000000}{\texttt{\ }}@><@\textcolor[HTML]{DE6F10}{\texttt{0.5}}@><@\textcolor[HTML]{1041FF}{\texttt{,}}@>
<@\textcolor[HTML]{000000}{\texttt{\ \ }}@><@\textcolor[HTML]{255CFF}{\texttt{"wrong\_values"}}@><@\textcolor[HTML]{1041FF}{\texttt{:}}@><@\textcolor[HTML]{000000}{\texttt{\ }}@><@\textcolor[HTML]{DE6F10}{\texttt{2}}@><@\textcolor[HTML]{1041FF}{\texttt{,}}@>
<@\textcolor[HTML]{000000}{\texttt{\ \ }}@><@\textcolor[HTML]{255CFF}{\texttt{"missing\_objects\_override"}}@><@\textcolor[HTML]{1041FF}{\texttt{:}}@><@\textcolor[HTML]{000000}{\texttt{\ }}@><@\textcolor[HTML]{FF1010}{\texttt{[}}@>
<@\textcolor[HTML]{000000}{\texttt{\ \ \ \ }}@><@\textcolor[HTML]{FF1010}{\texttt{[}}@><@\textcolor[HTML]{418310}{\texttt{"person"}}@><@\textcolor[HTML]{1041FF}{\texttt{,}}@><@\textcolor[HTML]{000000}{\texttt{\ }}@><@\textcolor[HTML]{DE6F10}{\texttt{5}}@><@\textcolor[HTML]{FF1010}{\texttt{]}}@><@\textcolor[HTML]{1041FF}{\texttt{,}}@>
<@\textcolor[HTML]{000000}{\texttt{\ \ \ \ }}@><@\textcolor[HTML]{FF1010}{\texttt{[}}@><@\textcolor[HTML]{418310}{\texttt{"animal"}}@><@\textcolor[HTML]{1041FF}{\texttt{,}}@><@\textcolor[HTML]{000000}{\texttt{\ }}@><@\textcolor[HTML]{DE6F10}{\texttt{4}}@><@\textcolor[HTML]{FF1010}{\texttt{]}}@><@\textcolor[HTML]{1041FF}{\texttt{,}}@>
<@\textcolor[HTML]{000000}{\texttt{\ \ \ \ }}@><@\textcolor[HTML]{FF1010}{\texttt{[}}@><@\textcolor[HTML]{418310}{\texttt{"environment"}}@><@\textcolor[HTML]{1041FF}{\texttt{,}}@><@\textcolor[HTML]{000000}{\texttt{\ }}@><@\textcolor[HTML]{DE6F10}{\texttt{1}}@><@\textcolor[HTML]{FF1010}{\texttt{]}}@>
<@\textcolor[HTML]{000000}{\texttt{\ \ }}@><@\textcolor[HTML]{FF1010}{\texttt{]}}@><@\textcolor[HTML]{1041FF}{\texttt{,}}@>
<@\textcolor[HTML]{000000}{\texttt{\ \ }}@><@\textcolor[HTML]{255CFF}{\texttt{"missing\_attributes\_override"}}@><@\textcolor[HTML]{1041FF}{\texttt{:}}@><@\textcolor[HTML]{000000}{\texttt{\ }}@><@\textcolor[HTML]{FF1010}{\texttt{[}}@>
<@\textcolor[HTML]{000000}{\texttt{\ \ \ \ }}@><@\textcolor[HTML]{FF1010}{\texttt{[}}@><@\textcolor[HTML]{418310}{\texttt{"action"}}@><@\textcolor[HTML]{1041FF}{\texttt{,}}@><@\textcolor[HTML]{000000}{\texttt{\ }}@><@\textcolor[HTML]{DE6F10}{\texttt{1.5}}@><@\textcolor[HTML]{FF1010}{\texttt{]}}@><@\textcolor[HTML]{1041FF}{\texttt{,}}@>
<@\textcolor[HTML]{000000}{\texttt{\ \ \ \ }}@><@\textcolor[HTML]{FF1010}{\texttt{[}}@><@\textcolor[HTML]{418310}{\texttt{"state"}}@><@\textcolor[HTML]{1041FF}{\texttt{,}}@><@\textcolor[HTML]{000000}{\texttt{\ }}@><@\textcolor[HTML]{DE6F10}{\texttt{0.5}}@><@\textcolor[HTML]{FF1010}{\texttt{]}}@>
<@\textcolor[HTML]{000000}{\texttt{\ \ }}@><@\textcolor[HTML]{FF1010}{\texttt{]}}@><@\textcolor[HTML]{1041FF}{\texttt{,}}@>
<@\textcolor[HTML]{000000}{\texttt{\ \ }}@><@\textcolor[HTML]{255CFF}{\texttt{"missing\_triplets\_override"}}@><@\textcolor[HTML]{1041FF}{\texttt{:}}@><@\textcolor[HTML]{000000}{\texttt{\ }}@><@\textcolor[HTML]{FF1010}{\texttt{[}}@><@\textcolor[HTML]{FF1010}{\texttt{[}}@><@\textcolor[HTML]{418310}{\texttt{"falling\ into"}}@><@\textcolor[HTML]{1041FF}{\texttt{,}}@><@\textcolor[HTML]{000000}{\texttt{\ }}@><@\textcolor[HTML]{DE6F10}{\texttt{5}}@><@\textcolor[HTML]{FF1010}{\texttt{]}}@><@\textcolor[HTML]{FF1010}{\texttt{]}}@><@\textcolor[HTML]{1041FF}{\texttt{,}}@>
<@\textcolor[HTML]{000000}{\texttt{\ \ }}@><@\textcolor[HTML]{255CFF}{\texttt{"wrong\_values\_override"}}@><@\textcolor[HTML]{1041FF}{\texttt{:}}@><@\textcolor[HTML]{000000}{\texttt{\ }}@><@\textcolor[HTML]{FF1010}{\texttt{[}}@>
<@\textcolor[HTML]{000000}{\texttt{\ \ \ \ }}@><@\textcolor[HTML]{FF1010}{\texttt{\{}}@>
<@\textcolor[HTML]{000000}{\texttt{\ \ \ \ \ \ }}@><@\textcolor[HTML]{255CFF}{\texttt{"attribute"}}@><@\textcolor[HTML]{1041FF}{\texttt{:}}@><@\textcolor[HTML]{000000}{\texttt{\ }}@><@\textcolor[HTML]{418310}{\texttt{"color"}}@><@\textcolor[HTML]{1041FF}{\texttt{,}}@>
<@\textcolor[HTML]{000000}{\texttt{\ \ \ \ \ \ }}@><@\textcolor[HTML]{255CFF}{\texttt{"default"}}@><@\textcolor[HTML]{1041FF}{\texttt{:}}@><@\textcolor[HTML]{000000}{\texttt{\ }}@><@\textcolor[HTML]{DE6F10}{\texttt{0.5}}@><@\textcolor[HTML]{1041FF}{\texttt{,}}@>
<@\textcolor[HTML]{000000}{\texttt{\ \ \ \ \ \ }}@><@\textcolor[HTML]{255CFF}{\texttt{"overrides"}}@><@\textcolor[HTML]{1041FF}{\texttt{:}}@><@\textcolor[HTML]{000000}{\texttt{\ }}@><@\textcolor[HTML]{FF1010}{\texttt{[}}@>
<@\textcolor[HTML]{000000}{\texttt{\ \ \ \ \ \ \ \ }}@><@\textcolor[HTML]{FF1010}{\texttt{[}}@><@\textcolor[HTML]{FF1010}{\texttt{[}}@><@\textcolor[HTML]{418310}{\texttt{"white"}}@><@\textcolor[HTML]{1041FF}{\texttt{,}}@><@\textcolor[HTML]{000000}{\texttt{\ }}@><@\textcolor[HTML]{418310}{\texttt{"yellow"}}@><@\textcolor[HTML]{1041FF}{\texttt{,}}@><@\textcolor[HTML]{000000}{\texttt{\ }}@><@\textcolor[HTML]{418310}{\texttt{"pink"}}@><@\textcolor[HTML]{FF1010}{\texttt{]}}@><@\textcolor[HTML]{1041FF}{\texttt{,}}@><@\textcolor[HTML]{000000}{\texttt{\ }}@><@\textcolor[HTML]{DE6F10}{\texttt{0.3}}@><@\textcolor[HTML]{FF1010}{\texttt{]}}@><@\textcolor[HTML]{1041FF}{\texttt{,}}@>
<@\textcolor[HTML]{000000}{\texttt{\ \ \ \ \ \ \ \ }}@><@\textcolor[HTML]{FF1010}{\texttt{[}}@><@\textcolor[HTML]{FF1010}{\texttt{[}}@><@\textcolor[HTML]{418310}{\texttt{"red"}}@><@\textcolor[HTML]{1041FF}{\texttt{,}}@><@\textcolor[HTML]{000000}{\texttt{\ }}@><@\textcolor[HTML]{418310}{\texttt{"blue"}}@><@\textcolor[HTML]{1041FF}{\texttt{,}}@><@\textcolor[HTML]{000000}{\texttt{\ }}@><@\textcolor[HTML]{418310}{\texttt{"green"}}@><@\textcolor[HTML]{1041FF}{\texttt{,}}@><@\textcolor[HTML]{000000}{\texttt{\ }}@><@\textcolor[HTML]{418310}{\texttt{"yellow"}}@><@\textcolor[HTML]{FF1010}{\texttt{]}}@><@\textcolor[HTML]{1041FF}{\texttt{,}}@><@\textcolor[HTML]{000000}{\texttt{\ }}@><@\textcolor[HTML]{DE6F10}{\texttt{0.8}}@><@\textcolor[HTML]{FF1010}{\texttt{]}}@>
<@\textcolor[HTML]{000000}{\texttt{\ \ \ \ \ \ }}@><@\textcolor[HTML]{FF1010}{\texttt{]}}@>
<@\textcolor[HTML]{000000}{\texttt{\ \ \ \ }}@><@\textcolor[HTML]{FF1010}{\texttt{\}}}@><@\textcolor[HTML]{1041FF}{\texttt{,}}@>
<@\textcolor[HTML]{000000}{\texttt{\ \ \ \ }}@><@\textcolor[HTML]{FF1010}{\texttt{\{}}@>
<@\textcolor[HTML]{000000}{\texttt{\ \ \ \ \ \ }}@><@\textcolor[HTML]{255CFF}{\texttt{"attribute"}}@><@\textcolor[HTML]{1041FF}{\texttt{:}}@><@\textcolor[HTML]{000000}{\texttt{\ }}@><@\textcolor[HTML]{418310}{\texttt{"action"}}@><@\textcolor[HTML]{1041FF}{\texttt{,}}@>
<@\textcolor[HTML]{000000}{\texttt{\ \ \ \ \ \ }}@><@\textcolor[HTML]{255CFF}{\texttt{"default"}}@><@\textcolor[HTML]{1041FF}{\texttt{:}}@><@\textcolor[HTML]{000000}{\texttt{\ }}@><@\textcolor[HTML]{DE6F10}{\texttt{1.5}}@><@\textcolor[HTML]{1041FF}{\texttt{,}}@>
<@\textcolor[HTML]{000000}{\texttt{\ \ \ \ \ \ }}@><@\textcolor[HTML]{255CFF}{\texttt{"overrides"}}@><@\textcolor[HTML]{1041FF}{\texttt{:}}@><@\textcolor[HTML]{000000}{\texttt{\ }}@><@\textcolor[HTML]{FF1010}{\texttt{[}}@>
<@\textcolor[HTML]{000000}{\texttt{\ \ \ \ \ \ \ \ }}@><@\textcolor[HTML]{FF1010}{\texttt{[}}@><@\textcolor[HTML]{FF1010}{\texttt{[}}@><@\textcolor[HTML]{418310}{\texttt{"sitting"}}@><@\textcolor[HTML]{1041FF}{\texttt{,}}@><@\textcolor[HTML]{000000}{\texttt{\ }}@><@\textcolor[HTML]{418310}{\texttt{"reading"}}@><@\textcolor[HTML]{FF1010}{\texttt{]}}@><@\textcolor[HTML]{1041FF}{\texttt{,}}@><@\textcolor[HTML]{000000}{\texttt{\ }}@><@\textcolor[HTML]{DE6F10}{\texttt{0.8}}@><@\textcolor[HTML]{FF1010}{\texttt{]}}@><@\textcolor[HTML]{1041FF}{\texttt{,}}@>
<@\textcolor[HTML]{000000}{\texttt{\ \ \ \ \ \ \ \ }}@><@\textcolor[HTML]{FF1010}{\texttt{[}}@><@\textcolor[HTML]{FF1010}{\texttt{[}}@><@\textcolor[HTML]{418310}{\texttt{"fishing"}}@><@\textcolor[HTML]{1041FF}{\texttt{,}}@><@\textcolor[HTML]{000000}{\texttt{\ }}@><@\textcolor[HTML]{418310}{\texttt{"sitting"}}@><@\textcolor[HTML]{FF1010}{\texttt{]}}@><@\textcolor[HTML]{1041FF}{\texttt{,}}@><@\textcolor[HTML]{000000}{\texttt{\ }}@><@\textcolor[HTML]{DE6F10}{\texttt{0.7}}@><@\textcolor[HTML]{FF1010}{\texttt{]}}@>
<@\textcolor[HTML]{000000}{\texttt{\ \ \ \ \ \ }}@><@\textcolor[HTML]{FF1010}{\texttt{]}}@>
<@\textcolor[HTML]{000000}{\texttt{\ \ \ \ }}@><@\textcolor[HTML]{FF1010}{\texttt{\}}}@>
<@\textcolor[HTML]{000000}{\texttt{\ \ }}@><@\textcolor[HTML]{FF1010}{\texttt{]}}@>
<@\textcolor[HTML]{FF1010}{\texttt{\}}}@>

\end{lstlisting}


V Tabulce~\ref{tab:exmple_extracts} je několik vět z prvního přepisu, k nimž je vypsána množina extrahovaných objektů $\mathcal{O}$,
množina extrahovaných atributů $\mathcal A$ a množina extrahovaných vazeb $\mathcal V$.

\begin{table}[H]
	\centering

	\def\objone{\begin{tabular}{l}
			water  \\[-1mm]
			people \\[-1mm]
			animals
		\end{tabular}}

	\def\objtwo{\begin{tabular}{l}
			woman   \\[-1mm]
			lounger \\[-1mm]
			book    \\[-1mm]
			parasol
		\end{tabular}}

	\def\objthree{\begin{tabular}{l}
			blanket \\[-1mm]
			basket
		\end{tabular}}

	\def\objfour{\begin{tabular}{l}
			girl         \\[-1mm]
			swimsuit \#1 \\[-1mm]
			boy          \\[-1mm]
			cap          \\[-1mm]
			tshirt       \\[-1mm]
			shorts
		\end{tabular}}

	\def\attrone{\begin{tabular}{l}
			people: count = 5
		\end{tabular}}

	\def\attrtwo{---}

	\def\attrthree{---}

	\def\attrfour{\begin{tabular}{l}
			girl: hairstyle = ponytail \\[-1mm]
			tshirt: pattern = striped
		\end{tabular}}

	\def\tripone{\begin{tabular}{l}
			---
		\end{tabular}}
	\def\triptwo{woman reading book}
	\def\tripthree{---}
	\def\tripfour{\begin{tabular}{l}
			% girl \to\ wearing \to\ swimsuit \#1 \\[-1mm]
			% boy \to\ wearing \to\ cap
			girl wearing swimsuit \#1 \\[-1mm]
			boy wearing cap
		\end{tabular}}

	\def\senone{\emph{na břehu rybníka je pět osob a několik zvířat}}
	\def\sentwo{\emph{vlevo sedí žena na lehátku čte si knihu pod slunečníkem}}
	\def\senthree{\emph{vedle sebe má rozprostřenou piknikovou deku s košíkem a s jídlem na piknik}}
	\def\senfour{\hspace{-1.5mm}\begin{tabular}{l}
			\emph{je to děvče s culíkem a v plavkách a chlapec} \\[-1mm]
			\emph{s kšiltovkou a pruhovanym tričkem a kraťasy}
		\end{tabular}}

	\begin{tabular}{|l|ccc|}
		\hline
		Věta 1            & \multicolumn{3}{l|}{\senone}                                                                                            \\ \hline
		\multirow{2}{*}{} & \multicolumn{1}{c|}{Objekty $\mathcal O_{1}$} & \multicolumn{1}{c|}{Atributy $\mathcal A_{1}$} & Vazby $\mathcal V_{1}$ \\ \cline{2-4}
		                  & \multicolumn{1}{l|}{\objone}                  & \multicolumn{1}{c|}{\attrone}                  & \tripone               \\ \hline
		\hline
		Věta 2            & \multicolumn{3}{l|}{\sentwo}                                                                                            \\ \hline
		\multirow{2}{*}{} & \multicolumn{1}{c|}{Objekty $\mathcal O_{2}$} & \multicolumn{1}{c|}{Atributy $\mathcal A_{2}$} & Vazby $\mathcal V_{2}$ \\ \cline{2-4}
		                  & \multicolumn{1}{l|}{\objtwo}                  & \multicolumn{1}{c|}{\attrtwo}                  & \triptwo               \\ \hline
		\hline
		Věta 3            & \multicolumn{3}{l|}{\senthree}                                                                                          \\ \hline
		\multirow{2}{*}{} & \multicolumn{1}{c|}{Objekty $\mathcal O_{3}$} & \multicolumn{1}{c|}{Atributy $\mathcal A_{3}$} & Vazby $\mathcal V_{3}$ \\ \cline{2-4}
		                  & \multicolumn{1}{l|}{\objthree}                & \multicolumn{1}{c|}{\attrthree}                & \tripthree             \\ \hline
		\hline
		Věta 4            & \multicolumn{3}{l|}{\senfour}                                                                                           \\ \hline
		\multirow{2}{*}{} & \multicolumn{1}{c|}{Objekty $\mathcal O_{4}$} & \multicolumn{1}{c|}{Atributy $\mathcal A_{4}$} & Vazby $\mathcal V_{4}$ \\ \cline{2-4}
		                  & \multicolumn{1}{l|}{\objfour}                 & \multicolumn{1}{c|}{\attrfour}                 & \tripfour              \\ \hline
	\end{tabular}
	\caption{Příklad konkrétních vět z nich extrahovaných entit}\label{tab:exmple_extracts}
\end{table}

Jak je možné na těchto ukázkách vidět, tak navržený systém je schopen z přirozeného textu extrahovat požadované sémantické informace.
Předpokladem pro kvalitní výstup je kvalitní referenční popis a SPGF gramatika.
Oba tyto faktory jsou zcela závislé na expertovi, který vytváří jak referenční popis, tak gramatiku.

To může představovat problém ve smyslu vysokých počátečních nákladů, protože tvorba referenčního popisu a SPGF gramatiky
vyžaduje znalost zde navrženého formátu a může být pro lidského experta časově náročná.
Naopak výhodou zvoleného přístupu je ale možnost upravovat a vylepšovat jak referenční popis, tak SPGF gramatiku i v budoucnu,
protože oba tyto vstupy jsou navržené tak, aby byly čitelné člověkem a bylo možné je v případě potřeby kdykoli upravit.

Dále tento expertní přístup a manuální správa vstupních dat umožňuje přesnou kontrolu, na rozdíl od jiných přístupů, například neuronových sítí, jejichž funkce je spíše black-box a
potřeba provést detailní změny nebo přímo kontrolovat některé aspekty procesu jsou velmi obtížné.

Celé přepisy použité pro další testování a získání ilustračních výstupů jsou následující:
\todo{má cenu vůbec řešit celkové výstupní ztráty, když stejně ztrátová tabulka má jenom ilustrační hodnoty?}
\todo{nechat ty přepisy přímo tady, nebo je dát jako přílohu?}
\begin{enumerate}
	\item \textbf{Přepis 1:} \\
	      \emph{Na břehu rybníka je pět osob a několik zvířat. Vlevo sedí žena na lehátku, čte si knihu
		      pod slunečníkem. Vedle sebe má rozprostřenou piknikovou deku s košíkem a s jídlem
		      na piknik. Opodál si hrají děti s míčem, hází si. Je to děvče s culíkem a v plavkách a
		      chlapec s kšiltovkou a pruhovanym tričkem a kraťasy. Na pravé straně sedí rybář,
		      který chytá ryby, má nahozený prut a vedle sebe podběrák. Před ním skáče žába
		      z kamene do rybníka. Mezi ženou a rybářem je uprostřed dítě, které vypadá, že padá
		      do rybníka, doufám, že ho někdo zachrání, že si ho někdo všimne. A vpředu vpravo
		      muž plave.}
	\item \textbf{Přepis 2:}\\
	      \emph{Vidím koupající se lidi. Pán plave kraula, malé dítě na břehu vypadá, že zakoplo a
		      zrovna padá do vody. Na kraji u rákosí vpravo sedí pán s čepicí, s kostkovanou košilí a
		      nějakou vestou chytá asi ryby. Na druhé straně, na levé straně obrázku je paní, která
		      pod deštníkem si čte nějakou zajímavou knížku, usmívá se. Na pozadí si děti házejí
		      míčem. Úplně vzádu vpravo nahoře běhá pejsek, kterej honí veverku, která před ním
		      leze, utíká na strom. Celkově svítí sluníčko, na nebi letí letadlo. Vidím tady čtyři stromy.
		      S tím, že na vodní hladině ještě se tady vyskytují zvířata jako je žába, kapr, kachna
		      s malými kachňátky. Pani vypadá, že má dovolenou, protože kromě toho, že si čte
		      knihu tak tam má ještě připravenou deku, nějaký piknik. Vypadá to, že má něco
		      dobrého.}
	\item \textbf{Přepis 3:} \\
	      \emph{Na obrázku vidím řeku a pobřeří, pobřeží řeky. V řece plavou kačeny. Matka s pěti
		      malými káčátky. Vyskakuje asi ryba. Je v ní plavec. Padá do ní dítě malé. Maminka
		      možná sedí na lehátku, čte si knížku, takže si ho asi moc nevšímá. Vedle maminky leží
		      deka. Na ní leží košík, je to jakýsi piknik. Má tam láhev s pitím, skleničku. Dítě tam má
		      nějaké dvě, dva kyblíčky, lopatičku a vypadá to, dítě má na sobě možná jednorázovou
		      plenu, že si asi hrálo s kyblíčkem a lopatičkou a teď nějakým záhadným způsobem
		      padá do vody. Upadlo, maminka se nedívá. Takže to může bejt asi problém. Vedle
		      sedí rybář, který se snaží lovit ryby. Vedle něho je kámen, z kterého skáče žába, asi ji
		      vyplašil. Rybář se usmívá, což je zvláštní, protože tam vedle něho padá to dítě do
		      vody, to mi přijde zvláštní. Potom za, v pozadí si hrají dvě takové už vzrostlejší děti, holka v plavkách, kluk
		      v tričko, šortky.}
	\item \textbf{Přepis 4 (pacient):}\\
	      \emph{Tady si děti házej míčem. Pes honí veverku. Rybář chytá ryby. Tajdle plave ňákej.
		      Tamhle si čte pani. Kačeny tajdle jsou. Jsou tam stromy, no a sluníčko. Tady dítě se
		      namáčí. Tam je žába, tajdle kapr. Tamhle letadlo, ptáci.}
	\item \textbf{Přepis 5 (pacient):}\\
	      \emph{Takže hrajou volejbal. Pes honí kočku. Tady se někdo opaluje. Tady si děcko hraje ve
		      vodě. Rybář. Tady plave někdo. Tady je ryba, kachna s káčaty. To je asi všechno. Žába.
		      Já bych řekla, že to je všechno. Tady ňákej pták na stromě. Svítí sluníčko. Letí letadlo.}
\end{enumerate}


Pro tyto přepisy bylo podle ztrátové tabulky (viz Výpis~\ref{lst:loss_table_example}) vypočteno hodnocení, jehož část (zkráceno kvůli délce) je vidět v Tabulce~\ref{tab:out_values}.
Kompletní výpisy je možné prozkoumat spolu se zdrojovými kódy online.
\todo{doplnit zbytek tabulky}
\begin{table}[ht!]
	\centering
	\begin{tabular}{|l|c|c|c|c|c|}
		\hline
		\multirow{2}{*}{\textbf{Typ ztráty}}                & \multicolumn{5}{c|}{\textbf{Číslo přepisu}}                                                     \\
		\cline{2-6}
		                                                    & \textbf{1}                                  & \textbf{2} & \textbf{3} & \textbf{4} & \textbf{5} \\
		\hline
		chybějící objekty                                   & 125.5                                       &            &            &            &            \\
		chybějící atributy                                  & 79.5                                        &            &            &            &            \\
		chybějící triplety                                  & 164.0                                       &            &            &            &            \\
		atributy s chybnou hodnotou                         & 2.0                                         &            &            &            &            \\
		\hline
		chybějící objekt s tagem \enquote{animal}           & 52.0                                        &            &            &            &            \\
		chybějící objekt s tagem \enquote{item}             & 24.0                                        &            &            &            &            \\
		chybějící objekt s tagem \enquote{person}           & 5.5                                         &            &            &            &            \\
		chybějící objekt s tagem \enquote{clothing}         & 24.0                                        &            &            &            &            \\
		\hline
		chybějící atributy typu \enquote{action}            & 5.0                                         &            &            &            &            \\
		chybějící atributy typu \enquote{color}             & 61.0                                        &            &            &            &            \\
		chybějící atributy typu \enquote{facial expression} & 0.5                                         &            &            &            &            \\
		chybějící atributy typu \enquote{hairstyle}         & 1.0                                         &            &            &            &            \\
		chybějící atributy typu \enquote{pattern}           & 1.0                                         &            &            &            &            \\
		\hline
		chybějící vazby typu \enquote{climbing}             & 2.0                                         &            &            &            &            \\
		chybějící vazby typu \enquote{catching}             & 4.0                                         &            &            &            &            \\
		chybějící vazby typu \enquote{sitting under}        & 2.0                                         &            &            &            &            \\
		chybějící vazby typu \enquote{swimming in}          & 2.0                                         &            &            &            &            \\
		chybějící vazby typu \enquote{watching}             & 2.0                                         &            &            &            &            \\
		chybějící vazby typu \enquote{chasing}              & 2.0                                         &            &            &            &            \\
		chybějící vazby typu \enquote{sitting on}           & 8.0                                         &            &            &            &            \\
		chybějící vazby typu \enquote{falling into}         & 0.0                                         &            &            &            &            \\
		chybějící vazby typu \enquote{running away from}    & 2.0                                         &            &            &            &            \\
		chybějící vazby typu \enquote{wearing}              & 20.0                                        &            &            &            &            \\
		chybějící vazby typu \enquote{throwing}             & 2.0                                         &            &            &            &            \\
		\hline
		chybné hodnoty atributů typu \enquote{action}       & 0.0                                         &            &            &            &            \\
		chybné hodnoty atributů typu \enquote{color}        & 0.0                                         &            &            &            &            \\
		chybné hodnoty atributů typu \enquote{pattern}      & 0.0                                         &            &            &            &            \\
		chybné hodnoty atributů typu \enquote{count}        & 2.0                                         &            &            &            &            \\
		chybné hodnoty atributů typu \enquote{hairstyle}    & 0.0                                         &            &            &            &            \\
		\hline
	\end{tabular}
	\caption{Ztrátové hodnoty testovaných přepisů (zkráceno)}\label{tab:out_values}
\end{table}
