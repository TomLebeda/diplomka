\subsection{Sémantika a porozumění řeči}
{\color{blue}co to je sémantika}

Sémantika je vědní obor související s lingvistikou a logikou a zabývá se zkoumáním významu.
Jejím předmětem je zkoumání významu frází, slov, vět a nebo i obecně jiných symbolů, pomocí kterých lze předávat informace.

Úloha porozumění řeči (angl.~spoken language understanding, SLU) je jednou z klíčových částí


{\color{blue}rozdíl sémantika vs syntaxe}

{\color{blue}sémantická analýza v AI vs v přirozené lingvistice}

{\color{blue}kde se sémantická analýza používá, k čemu je dobrá}

{\color{blue}metody SLU, extra potom gramatiky}
