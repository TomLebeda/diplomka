\subsection{Extrakce sémantické informace}\label{subsec:extrakce_semanticke_informace}
Dalším klíčovým bodem bylo najít způsob, jak porovnat uživatelský popis obrázku s jeho referenčním popisem.
To je úlohou subsystému pro získání sémantické informace z textu.
Cílem je z přirozené řeči extrahovat informace v takové podobě, aby je bylo možné porovnat se strukturovaným referenčním popisem.

% Dalším klíčovým bodem při návrhu systému pro hodnocení popisu obrázku bylo najít způsob, jak z získat požadovanou sémantickou informaci.
% Cílem je z přirozené lidské řeči extrahovat informace v takové podobě, aby je bylo možné porovnat s naším referenčním popisem obrázku a
% následně vyhodnotit jejich podobnost.

Extrakce sémantiky z přirozené řeči se běžně dělá z textového přepisu dané promluvy.
I v této práci tedy extrakce sémantických informací probíhá z textu.
To znamená, že pokud uživatel popíše obrázek mluvenou řečí, tak je potřeba promluvu převést do textu pomocí nějakého systému pro rozpoznání řeči.
Problematika rozpoznání řeči a převodu audia do textu je nad rámec této práce a bude v praxi řešena nějakou již existující implementací.
Ve zbytku práce bude tedy pro zjednodušení rovnou uvažovaným vstupem text.

Otázka extrakce sémantické informace z popisu přirozenou řečí se tedy zužuje na extrakci sémantiky z textu.
Jako způsob řešení byl zvolen přístup založený na sémantickém parsování pomocí bezkontextových gramatik.

Tento přístup funguje tak, že na základě referenčního popisu obrázku bude expertem sestavena gramatika,
podle které budou v textu detekované jednotlivé části sémantické informace.
Gramatika je v tomto kontextu sada pravidel, která definují, jaké promluvy jsou v textu očekávané a jak mohou tyto promluvy vypadat.
Program pak tato pravidla načte a prochází podle nich vstupní text a hledá, zda nějaké části textu \enquote{pasují} na dané pravidlo.
Samotný nalezený kus textu, který \enquote{pasuje} na nějaké pravidlo, ale ještě nemá žádný užitečný význam.
Proto pravidla obsahují zároveň i informaci o tom, jaká je struktura nalezené části textu a jaký význam mají jednotlivé části této struktury.
V českém jazyce by se jednalo o obdobu větného rozboru, kdy se například určuje, která část věty je podmět a která přísudek.

Strukturovaná a označkovaná část textu s doplněnou sémantickou informací je pak označována jako \emph{sémantická entita}.
Konkrétní syntaxe, použití a funkčnost těchto gramatik bude popsán později spolu s implementací v části~\ref{subsec:moje_gramatiky}.
Množina všech těchto sémantických entit, které byly ze vstupního textu extrahované, jsou pak použité pro vytvoření testovaného popisu.

\subsubsection{Podoba sémantických entit}
První otázkou, kterou bylo potřeba vyřešit pro získání sémantické informace z přirozeného popisu, byla její podoba.
Jinými slovy, jak by měla extrahovaná sémantika vypadat, aby ji bylo možné porovnat s referenčním popisem obrázku.

Vzhledem k tomu, že výše definovaný referenční popis (viz sekce~\ref{subsec:reprezentace_znalosti}) se skládá z objektů, jejich hierarchie, atributů a vazeb,
tak se nabízí přímo použít tyto typy informací.
Bylo tedy rozhodnuto, že z přepisu přirozené řeči budou extrahované objekty, jejich atributy a vazby mezi nimi.

\subsubsection{Extrakce sémantiky - objekty}
Základním typem informace, kterou musí být systém schopen z přepisu řeči získat, jsou samotné objekty, které expert definoval v referenčním popisu obrázku.

Například ze vstupní promluvy
\begin{center}
	\enquote{\emph{Na obrázku vidím psa s veverkou, strom a nějakou houbu.}}
\end{center}
by měl systém vrátit množinu detekovaných objektů $\mathcal{O}$:
\[
	\mathcal O = \bigl\{\, \texttt{pes}, \texttt{veverka}, \texttt{strom}, \texttt{houba}\, \bigr\}
\]
Ve své nejjednodušší podobě by se mohlo jednat pouze o detekci nějakých klíčových slov, které odpovídají názvům objektů.
Skutečná realizace je poněkud složitější, bere v potaz různá synonyma, tvary slov a alternativní vyjádření.
% Stejně jako zbytek SLU subsystému je založená na pravidlech bezkontextové gramatiky.
% Pomocí těchto pravidel lze specifikovat, jaké jsou očekávané vyjádření jednotlivých objektů ve scéně.
Detailněji bude popsána později spolu s konkrétní implementací v kapitole~\ref{sec:implementace}.

% Složitost detekce objektů plyne především z toho, že český jazyk je velmi bohatý na synonyma a různé tvary slov.
% To znamená, že jeden objekt může být popsaný různými lidmi zcela odlišnými slovy a nebo i jeden člověk může stejnou věc vyjádřit pokaždé jinak.
% Také záleží na kontextu daného slova, který může ovlivnit jeho konkrétní tvar.

Detekce samotných objektů by mohla být pro některé aplikace dokonce postačující sama o sobě.
Mohlo by se jednat o úlohy, kde je hlavním předmětem pouze zjistit, kolik objektů na obrázku člověk popíše, případně které to jsou.
Typickými příklady by mohli být nějaké testy paměti, pozornosti nebo jednoduché klasifikace.

\newpage
\subsubsection{Extrakce sémantiky - atributy}
Druhým typem sémantické informace, kterou je potřeba, aby byl systém schopen najít a extrahovat z textu, jsou atributy popisující vlastnosti objektů.

Ve srovnání s detekcí samotných objektů se jedná o značně složitější problém, protože pro extrakci atributu je potřeba mít informace o objektu,
na který se atribut váže, o názvu daného atributu a také o jeho přiřazené hodnotě.

Například ve větě
\begin{center}
	\enquote{\emph{Na obrázku vidím kluka v modrém tričku.}}
\end{center}
by měl systém detekovat, že objekt \enquote{\texttt{tričko}} má atribut \enquote{\texttt{barva}} s hodnotou \enquote{\texttt{modrá}}.

Kromě toho, že se tato informace skládá z více nezávislých částí, tak je možné si všimnout, že se ve zdrojové větě nikde nevyskytuje slovo \enquote{\emph{barva}}.
Toto je tedy informace, kterou musí být systém schopen nějakým způsobem indukovat z okolních dat a referenčního popisu.

Dále by jiný uživatel mohl popsat stejný obrázek třeba větou
\begin{center}
	\enquote{\emph{Vidím nějakého kluka v tričku, které je modré}}.
\end{center}
Tato promluva obsahuje stejný objekt, atribut i hodnotu, ale vyjádřenou ve zcela jiné podobě.
Jak je tedy zřejmé, zde již není možné použít pouze nějakou formu detekce klíčových slov, ale bude potřeba komplikovanějšího přístupu.

Právě dříve zmíněná pravidla, která udávají různé formy hledané informace, umožňují zachytit i takovéto složitější struktury v různých formách.
Konkrétní realizace bude opět popsána později, viz sekce~\ref{sec:implementace}.

\newpage
\subsubsection{Extrakce sémantiky - vazby mezi objekty}
Posledním typem informace, kterou je potřeba získat z textu, jsou vazby mezi objekty.

Například pro větu
\begin{center}
	\enquote{\emph{Na obrázku hnědého vidím psa, co běží za oranžovou veverkou.}}
\end{center}
je třeba, aby systém dokázal detekovat vazbu mezi objektem psa a veverky:
\begin{center}
	\begin{tikzpicture}
		\node[draw, rounded corners=3mm, minimum height=7mm, minimum width=1cm] (n1) at (0, 0) {\texttt{pes\vphantom{k}}};
		\node[draw, rounded corners=3mm, minimum height=7mm, minimum width=1.8cm] (n2) at (4, 0) {\texttt{veverka\vphantom{p}}};
		\draw[-Stealth] (n1) -- node[midway, above] {\texttt{běží za}} (n2);
	\end{tikzpicture}
\end{center}

Jak je zřejmé, pro sestavení vazby je - stejně jako pro atributy - potřeba tří částí informace: zdrojový objekt, cílový objekt a název vazby.
Na rozdíl od detekce atributů ale představuje detekce vazeb unikátní problém, protože zdrojový a cílový objekt jsou stejného typu - oba jsou to objekty.
To znamená, že v při konstrukci tripletů musí být použit nějaký mechanismus, který rozpozná, který z objektů má být cílový a který zdrojový.
Tento problém je řešen také v rámci definice pravidel v bezkontextové gramatice pomocí tagů, detailní popis v sekci~\ref{subsec:moje_gramatiky}.

Je vhodné zmínit, že ačkoli byly v kapitole~\ref{subsec:reprezentace_znalosti} definované čtyři aspekty popisu,
nyní jsou řešené pouze tři různé typy sémantických entit.
To je proto, že hierarchie mezi objekty byla při návrhu považována za speciální případ vazby mezi objekty a nebyla pro ni vytvořena samostatná kategorie.
