\subsection{Extrakce sémantické informace}
Dalším klíčovým bodem při návrhu systému pro hodnocení popisu obrázku bylo najít způsob, jak z přirozené řeči extrahovat požadovanou sémantickou informaci.
Cílem je z přirozené lidské řeči získat informace v takové podobě, aby je bylo možné porovnat s naším referenčním popisem obrázku a
následně vyhodnotit jejich podobnost.

\todo{ozdrojovat?}
Extrakce sémantiky z přirozené řeči se běžně dělá z textového přepisu dané promluvy.
I v této práci tedy extrakce sémantických informací probíhá z textu.
To znamená, že pokud uživatel popíše obrázek mluvenou řečí, tak je potřeba promluvu převést do textu pomocí nějakého systému pro rozpoznání řeči (ASR).
\todo{existují metody co to dělají rovnou z audia?}
\todo{změna času? (z minulého do přítomného)}
Problematika rozpoznání řeči a převodu audia do textu je nad rámec této práce a předpokládá se, že bude v praxi řešena nějakou již existující implementací.
Ve zbytku práce bude tedy pro zjednodušení rovnou předpokládaným vstupem text.
\todo{doplnit, že jsem při vývoji používal ruční přepisy? Případně zmínit odkud jsem je vzal?}

Otázka extrakce sémantické informace se tedy zužuje na extrakci sémantiky z přirozené řeči v textové podobě.
Jako způsob řešení byl zvolen přístup založený na sémantickém parsování pomocí bezkontextových gramatik,
který je v souladu s volnou expertního přístupu v celé této práci.

Základní koncept celého sub-systému pro extrakci sémantiky byl navržen tak, že podle referenčního popisu obrázku bude expertem sestavena gramatika,
podle které budou v textu detekované jednotlivé sémantické entity.
Gramatika je v tomto kontextu sada pravidel, která definují, jaké promluvy jsou v textu očekávané.
Dále tato pravidla také udávají informace o tom, jakou sémantickou entitu daná detekovaná promluva vyjadřuje.
Konkrétní syntaxe, použití a funkčnost těchto gramatik bude popsán později spolu s implementací v části~\ref{subsec:moje_gramatiky}.

Tyto sémantické entity pak budou porovnané s referenčním popisem

\subsubsection{Sémantické entity}
První otázkou, kterou bylo potřeba vyřešit pro získání sémantické informace z přirozeného popisu, byla její podoba.
Jinými slovy, jak by měla extrahovaná sémantika vypadat, aby ji bylo možné porovnat s referenčním popisem obrázku.

Vzhledem k tomu, že výše definovaný referenční popis (viz sekce~\ref{subsec:reprezentace_znalosti}) se skládá z objektů, jejich hierarchie, atributů a vazeb,
tak se nabízí přímo tyto čtyři typy informací hledat v textu.

Bylo tedy rozhodnuto, že z přepisu přirozené řeči budou extrahované:
\begin{itemize}
	\item objekty \to systém pro extrakci sémantiky byl měl být schopen v textu detekovat všechny zmíněné objekty, které jsou zároveň definované v referenčním popisu.
	\item atributy/vlastnosti objektů
	\item vazby/vztahy mezi objekty
\end{itemize}
