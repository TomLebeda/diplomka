\subsection{Obecná architektura systému}
Obecná architektura celého systému vychází z jeho požadované funkčnosti,
kterou je porovnání obrázku s jeho popisem v přirozené řeči, a to na sémantické úrovni.
Z toho pak plyne, že celý systém se ve své podstatě skládá ze tří základních částí:
\begin{enumerate}
	\item referenční (vzorový) popis daného obrázku
	\item sub-systém pro zpracování přirozené řeči a tvorbu testovaného popisu
	\item porovnání referenčního a testovaného popisu
\end{enumerate}

Jednotlivé části spolu vzájemně fungují následujícím způsobem:
Uživateli je prezentován obrázek a jeho úkolem je popsat, co na obrázku vidí.
Získaný popis v přirozené řeči je převeden na text pomocí systému pro rozpoznání řeči (angl.~automatic speech recognition, ASR).
Z tohoto přepisu je extrahována sémantická informace, ze které je vytvořen testovaný popis.
Testovaný popis je porovnaný s referenčním (vzorovým) popisem daného obrázku.
Výsledek tohoto porovnání lze pak považovat za finální výstup, ale také je možné jej použít jako vstup pro další zpracování
(např.~vektor příznaků pro klasifikátor).
Schématické znázornění je na Obrázku~\ref{fig:architecture_scheme}.

\begin{figure}[ht!]
	\centering
	\begin{tikzpicture}[minimum width=2.5cm, minimum height=1cm]
		\footnotesize
		\node[anchor=south, inner sep=4pt, minimum height=0cm] (img) at (0,0) {\texttt{obrázek}};
		\node[draw] (user) at (-3,-1.5) {\texttt{UŽIVATEL}};
		\node[draw] (exp) at (3,-3.5) {\texttt{EXPERT}};
		\node[draw] (asr) at (-3,-3.5) {\texttt{ASR + SLU}};
		\node[draw] (comp) at (0,-5) {\texttt{KOMPARÁTOR}};
		\draw[-Stealth,rounded corners=2mm] (0,0) -- (0,-0.4) -| (exp.north);
		\draw[-Stealth,rounded corners=2mm] (0,0) -- (0,-0.4) -| (user.north);
		\draw[-Stealth] (user) -- node[minimum width=0cm,anchor=west,midway] {\texttt{řeč}} (asr);
		\draw[-Stealth,rounded corners=2mm] (asr) |- node[minimum width=0cm,left, near start] {\parbox{2cm}{\texttt{testovaný}\\[-3mm] \texttt{\phantom{llll}popis}}} (comp.west);
		\draw[-Stealth,rounded corners=2mm] (exp) |- node[minimum width=0cm,right, near start] {\parbox{2cm}{\texttt{referenční}\\[-3mm] \texttt{popis}}} (comp.east);
		\draw[-Stealth] (comp.south) -- node[midway, right, minimum width=0cm]{\texttt{hodnocení}} ++(0, -1);
	\end{tikzpicture}
	\caption{Schéma obecné architektury systému}\label{fig:architecture_scheme}
\end{figure}

Tato práce řeší problematiku srovnání obrázku s jeho popisem v přirozeném jazyce na sémantické úrovni expertním přístupem.
Od toho se poté odvíjí konkrétní algoritmy, formáty a postupy navržené a implementované v této práci,
které jsou podrobněji popsány v pozdějších kapitolách.
Je ale vhodné zmínit, že během návrhu bylo dbáno na to, aby bylo možné pro reálná nasazení některé implementace v případě potřeby zaměnit nebo upravit,
aby lépe vyhovovaly specifickým požadavkům pro dané použití.

Jak je možné ze schématu na Obrázku~\ref{fig:architecture_scheme} vidět, lidský expert je potřebný pro vytvoření referenčního popisu obrázku
(a také pro další vstupní data pro komparátor, popsáno později).
Všechna tato data, která jsou závislá na lidském expertovi, je ovšem možné vytvořit pouze jednou v rámci přípravy
a poté je lze již bez nutnosti přítomnosti experta opakovaně používat.
