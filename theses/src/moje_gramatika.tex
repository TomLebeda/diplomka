\subsection{Sémantické parsování pomocí gramatik}\label{subsec:moje_gramatiky}
Extrakce sémantické informace byla jedním z hlavních problémů řešených v této práci.
Jak již bylo řečeno dříve, byl zvolen přístup založený na parsování textu pomocí sémantických bezkontextových gramatik.

Při prvotních experimentech byla použita již existující implementace tohoto konceptu,
\todo{referovat na SpeechCloud}
používající standard \texttt{SRGS}
\todo{referovat SRGS}
Během prvních experimentů ale bylo zjištěno, že funkčnost, kterou nabízí existující implementace, nebude postačovat.
Proto byl navržen a implementován nový formát odvozený právě ze \texttt{SRGS} standardu, který byl označen jako \texttt{SPGF} (Semantic Parsing Grammar Format).

Důvodem pro návrh a implementaci vlastního formátu gramatiky a k němu příslušejícímu systému byly
% chybí GARBAGE
% pouze greedy matching 
% -> přidal jsem 
