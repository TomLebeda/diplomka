\clearpage
\section{Úvod}
% NOTE: obecně oblast - co to je, k čemu se to používá
V současném výzkumu aplikované lingvistiky a počítačového zpracování přirozeného
jazyka zaujímá oblast zpracování textu, sémantická analýza a reprezentace znalostí význačnou pozici.
Tyto problematiky jsou nedílnou součástí mnoha běžně používaných systémů,
mezi které patří dnes již běžně dostupní hlasoví asistenti, automatické překlady a titulkování,
či systémy pro extrakci informací v různých průmyslových nasazeních.
% Ve všech těchto nasazeních jsou nějakou formou řešené také problémy sémantiky a reprezentace znalostí, 
% jakožto nedílné součásti těchto řešení.
% Kromě průmyslových a komerčních aplikacích se však zpracování přirozeného textu a sémantická analýza
% uplatňuje i v řadě dalších, více specializovaných oblastech, jako je například medicína,
% kde jsou tyto koncepty používané například pro automatické zpracování a extrakci informací.

% NOTE: co bude v této práci + motivace
Cílem této práce je navrhnout a implementovat systém pro určení míry korespondence obrázku a jeho popisu v přirozené řeči, a to na sémantické úrovni.
Motivací pro takovýto systém je možnost jeho využití pro vytvoření aplikace usnadňující vyšetření pro detekci kognitivních
poruch s využitím metod strojového učení.
Základní myšlenkou je využití předpokladu, že snížené kognitivní funkce se projeví mimo jiné
i na schopnosti člověka detailně popsat komplexní scénu zobrazenou na předloženém obrázku.

Hlavní koncept prezentovaný v této práci spočívá v extrahování sémantické informace obsažené v přirozeném popisu,
srovnání se vzorovým popisem a ohodnocení míry korespondence v podobě vektoru hodnot, který je možné
následně použít jako vstup pro další zpracování, například jako vektor příznaků pro klasifikaci.

% NOTE: Na co byl kladen důraz
Během návrhu celkového konceptu systému, jeho jednotlivých částí a i následné implementace, byl kladen důraz na to,
aby byly jednotlivé procesy \enquote{průhledné}, bylo možné detailně sledovat všechny kroky
a v případě potřeby dělat detailní změny či vylepšení, pro zdokonalení výsledků v konkrétním nasazení.
Dále bylo dbáno na modularitu systému, aby bylo možné v případě potřeby jednotlivé části nahradit za případné alternativy,
nebo doplnit dodatečnými rozšířeními.
Je vhodné poznamenat, že ačkoli bylo primární motivací již zmíněná podpora diagnostiky kognitivních poruch,
byl systém navrhován obecně tak, aby jej bylo možné použít i pro případná další nasazení benefitující ze zde navržených metod.

% NOTE: co je v práci popsáno
V práci bude popsán návrh jednotlivých částí i celkové architektury, včetně zdůvodnění učiněných rozhodnutí.
Následně bude popsána samotná realizace, společně s problémy a jejich řešením, které se v průběhu vypracování vyskytly.
