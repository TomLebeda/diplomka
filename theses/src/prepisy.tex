\begin{enumerate}
	\item \textbf{Přepis 1:} \emph{Na břehu rybníka je pět osob a několik zvířat. Vlevo sedí žena na lehátku, čte si knihu
		      pod slunečníkem. Vedle sebe má rozprostřenou piknikovou deku s košíkem a s jídlem
		      na piknik. Opodál si hrají děti s míčem, hází si. Je to děvče s culíkem a v plavkách a
		      chlapec s kšiltovkou a pruhovanym tričkem a kraťasy. Na pravé straně sedí rybář,
		      který chytá ryby, má nahozený prut a vedle sebe podběrák. Před ním skáče žába
		      z kamene do rybníka. Mezi ženou a rybářem je uprostřed dítě, které vypadá, že padá
		      do rybníka, doufám, že ho někdo zachrání, že si ho někdo všimne. A vpředu vpravo
		      muž plave.}
	\item \textbf{Přepis 2:} \emph{Vidím koupající se lidi. Pán plave kraula, malé dítě na břehu vypadá, že zakoplo a
		      zrovna padá do vody. Na kraji u rákosí vpravo sedí pán s čepicí, s kostkovanou košilí a
		      nějakou vestou chytá asi ryby. Na druhé straně, na levé straně obrázku je paní, která
		      pod deštníkem si čte nějakou zajímavou knížku, usmívá se. Na pozadí si děti házejí
		      míčem. Úplně vzádu vpravo nahoře běhá pejsek, kterej honí veverku, která před ním
		      leze, utíká na strom. Celkově svítí sluníčko, na nebi letí letadlo. Vidím tady čtyři stromy.
		      S tím, že na vodní hladině ještě se tady vyskytují zvířata jako je žába, kapr, kachna
		      s malými kachňátky. Pani vypadá, že má dovolenou, protože kromě toho, že si čte
		      knihu tak tam má ještě připravenou deku, nějaký piknik. Vypadá to, že má něco
		      dobrého.}
	\item \textbf{Přepis 3:} \emph{Na obrázku vidím řeku a pobřeří, pobřeží řeky. V řece plavou kačeny. Matka s pěti
		      malými káčátky. Vyskakuje asi ryba. Je v ní plavec. Padá do ní dítě malé. Maminka
		      možná sedí na lehátku, čte si knížku, takže si ho asi moc nevšímá. Vedle maminky leží
		      deka. Na ní leží košík, je to jakýsi piknik. Má tam láhev s pitím, skleničku. Dítě tam má
		      nějaké dvě, dva kyblíčky, lopatičku a vypadá to, dítě má na sobě možná jednorázovou
		      plenu, že si asi hrálo s kyblíčkem a lopatičkou a teď nějakým záhadným způsobem
		      padá do vody. Upadlo, maminka se nedívá. Takže to může bejt asi problém. Vedle
		      sedí rybář, který se snaží lovit ryby. Vedle něho je kámen, z kterého skáče žába, asi ji
		      vyplašil. Rybář se usmívá, což je zvláštní, protože tam vedle něho padá to dítě do
		      vody, to mi přijde zvláštní. Potom za, v pozadí si hrají dvě takové už vzrostlejší děti, holka v plavkách, kluk
		      v tričko, šortky.}
	\item \textbf{Přepis 4:} \emph{Tady si děti házej míčem. Pes honí veverku. Rybář chytá ryby. Tajdle plave ňákej.
		      Tamhle si čte pani. Kačeny tajdle jsou. Jsou tam stromy, no a sluníčko. Tady dítě se
		      namáčí. Tam je žába, tajdle kapr. Tamhle letadlo, ptáci.}
	\item \textbf{Přepis 5:} \emph{Takže hrajou volejbal. Pes honí kočku. Tady se někdo opaluje. Tady si děcko hraje ve
		      vodě. Rybář. Tady plave někdo. Tady je ryba, kachna s káčaty. To je asi všechno. Žába.
		      Já bych řekla, že to je všechno. Tady ňákej pták na stromě. Svítí sluníčko. Letí letadlo.}
\end{enumerate}
